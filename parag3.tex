\documentclass[12pt,a4paper,titlepage, oneside]{book}
\usepackage[utf8]{inputenc}
\usepackage[russian]{babel}
\usepackage[OT1]{fontenc}
\usepackage{amsmath}
\usepackage{amsthm}
\usepackage{mathrsfs}
\usepackage{indentfirst}
\usepackage{amsfonts}
\usepackage{amssymb}
\usepackage[left=2cm,right=2cm,top=2cm,bottom=2cm]{geometry}
\title{Функциональный анализ}

\newcommand{\overbar}[1]{\mkern 1.5mu\overline{\mkern-1.5mu#1\mkern-1.5mu}\mkern 1.5mu}

\theoremstyle{definition}
\newtheorem*{definition}{Определение}

\theoremstyle{plain}
\newtheorem*{theorem}{Теорема}

\theoremstyle{remark}
\newtheorem*{remark}{Замечание}

\theoremstyle{remark}
\newtheorem*{example}{Пример}

\theoremstyle{remark}
\newtheorem*{examples}{Примеры}

\theoremstyle{remark}
\newtheorem*{cexample}{Контр-пример}

\theoremstyle{plain}
\newtheorem*{lemma}{Лемма}

\theoremstyle{plain}
\newtheorem*{corollary}{Следствие}

\setcounter{tocdepth}{1}

\def\labelitemi{--}

\renewcommand{\qedsymbol}{\rule{0.7em}{0.7em}}

\begin{document}

\begin{titlepage}

\begin{center}
\vfill

Санкт-Петербургский государственный университет\\
\ \\

\vfill

{\large\bf ФУНКЦИОНАЛЬНЫЙ АНАЛИЗ\\}
\ \\
Лекции для студентов факультета ПМ-ПУ\\
(III курс, 6-ой семестр)

\vfill

\hfill\vbox
{
\hbox{Доцент кафедры моделирования электромеханических}
\hbox{и компьютерных систем, кандидат физ.-мат. наук}
\hbox{Владимир Олегович Сергеев}
}

\vfill

Санкт-Петербург, 2016
\end{center}

\end{titlepage}


\section{Теоремы Фредгольма}
Пусть $A$ вполне непрерывный оператор в пространстве Банаха $X$: $A \in \sigma(X,X)$. Рассмотрим уравнения второго рода с вполне непрерывными операторами $A$ и $A^*$:
\begin{equation}
(E - A)x=y , y \in X
\end{equation}
\begin{equation}
(E - A)z=\emptyset, z \in N(E-A)
\end{equation}
\begin{equation}
(I - A^*)f=\omega , f \in X^*, \omega \in X^* 
\end{equation}
\begin{equation}
(I - A^*)\psi=\emptyset, \psi \in N(I - A^*)
\end{equation}
\begin{theorem}[Первая теорема Фредгольма]
Следующие 4 утверждения эквивалентны:
\begin{enumerate}
\item Уравнение (1) имеет решение при любой правой части
\item Уравнение (2) имеет только тривиальное решение
\item Уравнение (3) имеет решение при любой правой части
\item Уравнение (4) имеет только тривиальное решение
\end{enumerate}
\end{theorem}
\begin{proof}
Докажем, например, что из 4 следует 1.\\
Пусть выполнено 4: $(I - A^*)=\emptyset$. Предположим противное : 1 не верно: $R(E - A)\neq X$. Пусть $y_0 \in X$, но $y_0 \not \in R(E - A)$. По теореме Хана-Банаха (следствие 2) существует линейный функционал $f_0 \in X^*$ такой, что $<y_0,f_0>=1$, $<y,f_0>=0$ для всех $y \in R(E-A)$. Тогда $<(E-A)x,f_0>=0$ для всех $x \in X$, $<x,(I-A^*f_0)>=0$ для всех $x \in X$.\\
Тогда $(I-A^*)f_0=\emptyset$, т.е. $f_0 \in N(I-A^*)$ и $f_0 \neq \emptyset$, что противоречит 4.
\end{proof}
\begin{theorem}[Вторая теорема Фредгольма]
Уравнения (2) и (4) имеют одинаковое конечное число линейно независимых решений.
\end{theorem}
\begin{theorem}[Третья теорема Фредгольма]
Для того, чтобы уравнение (1) имело решение, необходимо и достаточно, чтобы $<y,\psi>=0$ для любого решения $\psi$ уравнения (4).
\end{theorem}
\begin{proof}
\underline{Необходимость}. Если $N(E-A)=\emptyset$, то по второй теореме Фредгольма $N(I-A^*)=\emptyset$. Если же $N(E-A)\neq \emptyset$, то уравнение $(E-A)x=y_0$, $y_0 \neq \emptyset$ имеет решение $x_0$. Пусть $\psi \in N(I-A^*)$. Тогда
\begin{center}
$<y_0,\psi>=<(E-A)x_0,\psi>=<x_0,(I-A^*)\psi>=0$
\end{center}
\underline{Достаточность}. Пусть $<y_0,\psi>=0$ для всех $\psi \in N(I-A^*)$. \underline{Предположим}, что \\$y_0 \not \in R(E-A)$, т.е. решение уравнения $(E-A)x=y_0$ не существует. Так как по теореме Шаундера $R(E-A)$ есть подпространство, то существует линейный функционал $f\in X^*$, такой что 
\begin{center}
$<y_0,f>=1$ и $<(E-A)x,f>=0$
\end{center}
для любых элементов x,
\begin{center}
$<x,(I-A^*)f>=0$ и $(I-A^*)f=\emptyset$, $f \in N(I-A^*)$,
\end{center}
т.е. $f$ удовлетворяет уравнению (4). Таким образом,
\begin{center}
 $f \in N(I-A^*)$, $f\neq \emptyset$ и $<y_0,f>=1$.
\end{center}
Полученное противоречие показывает, что наше предположение неверно.
\end{proof}
\end{document}
