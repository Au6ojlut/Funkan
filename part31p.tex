\documentclass[12pt,a4paper,titlepage, oneside]{book}
\usepackage[utf8]{inputenc}
\usepackage[russian]{babel}
\usepackage[OT1]{fontenc}
\usepackage{amsmath}
\usepackage{amsthm}
\usepackage{mathrsfs}
\usepackage{indentfirst}
\usepackage{amsfonts}
\usepackage{amssymb}
\usepackage[left=2cm,right=2cm,top=2cm,bottom=2cm]{geometry}
\title{Функциональный анализ}

\newcommand{\overbar}[1]{\mkern 1.5mu\overline{\mkern-1.5mu#1\mkern-1.5mu}\mkern 1.5mu}

\theoremstyle{definition}
\newtheorem*{definition}{Определение}

\theoremstyle{plain}
\newtheorem*{theorem}{Теорема}

\theoremstyle{remark}
\newtheorem*{remark}{Замечание}

\theoremstyle{remark}
\newtheorem*{example}{Пример}

\theoremstyle{remark}
\newtheorem*{examples}{Примеры}

\theoremstyle{remark}
\newtheorem*{cexample}{Контр-пример}

\theoremstyle{plain}
\newtheorem*{lemma}{Лемма}

\theoremstyle{plain}
\newtheorem*{corollary}{Следствие}

\setcounter{tocdepth}{1}

\def\labelitemi{--}

\renewcommand{\qedsymbol}{\rule{0.7em}{0.7em}}

\begin{document}

\begin{titlepage}

\begin{center}
\vfill

Санкт-Петербургский государственный университет\\
\ \\

\vfill

{\large\bf ФУНКЦИОНАЛЬНЫЙ АНАЛИЗ\\}
\ \\
Лекции для студентов факультета ПМ-ПУ\\
(III курс, 6-ой семестр)

\vfill

\hfill\vbox
{
\hbox{Доцент кафедры моделирования электромеханических}
\hbox{и компьютерных систем, кандидат физ.-мат. наук}
\hbox{Владимир Олегович Сергеев}
}

\vfill

Санкт-Петербург, 2016
\end{center}

\end{titlepage}


Обозначим множество элементов пространства $H$, таких что ${\parallel x \parallel} _H \leq C$ через $K$. Пусть $\lbrace x_n \rbrace$ - произвольная последовательность элементов множества $K$.

\begin{enumerate}
\item Для всех элементов $x_n$ вычислим значения функционала $f_1$:

\begin{center}
$f_1(x)=<x,\Psi_1>=(x,\Psi_1)$.
\end{center}
Числовая последовательность $f_1(x_n)$ ограничена:
\begin{center}
$\vert f_1(x_n)\vert = \vert<x,\Psi_1>\vert \leq \parallel x_n\parallel \leq C$.
\end{center}
По теореме Больцано-Вейерштрасса в $\lbrace f_1(x_n) \rbrace$ существует сходящаяся подпоследовательность, обозначим её 
$<x_{N(1,K)},\Psi_1>$,
\begin{center}
$\lbrace x_{N(1,K)} \rbrace \subset
 \lbrace x_n \rbrace$, $<x_{N(1,K)},\Psi_1> \to \alpha_1$ при $K\to\infty$
\end{center}

\item Рассмотрим последовательность $\lbrace x_{N(1,K)} \rbrace$ и вычислим значения функционала $f_2$:
\begin{center}
$f_2(x)=<x,\Psi_2>$ 
\end{center}
на элементах этой последовательности. Числовая последовательность $ < x_{N(1,K)},\Psi_2>$ ограничена, существует её сходящаяся подпоследовательность. Обозначим её 
$<x_{N(2,K)},\Psi_2>$:
\begin{center}
$<x_{N(2,K)},\Psi_2> \to \alpha_2$ при
 $K\to\infty$, 
 \end{center}
 \begin{center}
$\lbrace x_{N(2,K)} \rbrace \subset 
\lbrace x_{N(1,K)} \rbrace \subset 
 \lbrace x_n \rbrace$,
  \end{center}
 \begin{center}
 $<x_{N(2,K)},\Psi_1> \to \alpha_1$ при
 $K\to\infty$. 
\end{center}
\end{enumerate}
Продолжая этот процесс, получим последовательности 
$\lbrace x_{N(i,K)} \rbrace$, $i=1,2,3...$, такие что 
 \begin{center}
 $\lbrace x_{N(1,K)} \rbrace \supset
\lbrace x_{N(2,K)} \rbrace \supset
\lbrace x_{N(3,K)} \rbrace \supset$...
\end{center}
и для которых
 \begin{center}
 $<x_{N(i,K)},\Psi_i> \to \alpha_i$ при
 $K\to\infty$, 
 \end{center}
 \begin{center}
 $<x_{N(i-1,K)},\Psi_{i-1}> \to \alpha_{i-1}$ при
 $K\to\infty$, 
 \end{center}
 \begin{center}
 $.........$
 \end{center}
 \begin{center}
 $<x_{N(1,K)},\Psi_{1}> \to \alpha_{1}$ при
 $K\to\infty$, 
 \end{center}
 Рассмотрим "диагональные элементы" $ x_{N(i,i)}$. Для них при любом фиксированном $K$ значения 
 $<x_{N(i,i)},\Psi_K>$ стремятся к $\alpha_K$ при 
$i \to \infty$.\\
Так как
\begin{equation*}
 \sum\limits_{k=1}^n 
{\vert<x_{N(i,i)},\Psi_K> \vert}^2 \leq C^2
\end{equation*}
  при любом $i$, то переходя к пределу при $i\to\infty$ получим $\sum\limits_{k=1}^n \alpha_k^2 \leq C^2$. Так как $n$ любое, то $\sum\limits_{k=1}^\infty \alpha_k^2 <+\infty$. Следовательно существует элемент 
\begin{equation*}
\widetilde{x}= \sum\limits_{k=1}^\infty 
\alpha_k\Psi_{k} \in K
\end{equation*}
Покажем, что последовательность
 $\lbrace x_{N(m,m)} \rbrace$
 слабо сходится к элементу $\widetilde{x}$.\\
 Рассмотрим произвольный элемент $y$ пространства 
 $H$. Он определяет функционал $f$:
 
\begin{equation*}
f(x)=(x,y)
\end{equation*}
Как элемент пространства $H$ элемент $y$ может быть представлен в виде 
\begin{equation*}
y=\sum\limits_{k=1}^\infty y_k\Psi_k =
\sum\limits_{k=1}^n y_k\Psi_k +
\sum\limits_{k=n+1}^\infty y_k\Psi_k
\end{equation*}
Для заданного $\epsilon$ найдём номер $n(\epsilon)$, такой что при $n>n(\epsilon)$
\begin{equation*}
\sum\limits_{k=n+1}^\infty y_k^2 < {\epsilon}^2 
\end{equation*}
\textbf{Зафиксируем значение $n$, $n>n(\epsilon)$.}\\

Значение функционала $f$ на элементе $x_{N(m,m)}-\widetilde{x}$ равно
\begin{equation*}
<x_{N(m,m)}-\widetilde{x},y>= 
<x_{N(m,m)}-\widetilde{x},\sum\limits_{k=1}^n y_k\Psi_k>+
<x_{N(m,m)}-\widetilde{x},\sum\limits_{k=n+1}^\infty y_k\Psi_k>
\end{equation*}
Во втором слагаемом значение
\begin{equation*}
\vert <x_{N(m,m)},\sum\limits_{k=n+1}^\infty y_k\Psi_k> \vert \leq
C\cdot\epsilon
\end{equation*}
и значение 
\begin{equation*}
\vert <\widetilde{x},\sum\limits_{k=n+1}^\infty y_k\Psi_k> \vert \leq
C\cdot\epsilon
\end{equation*}

В первом слагаемом значение
\begin{equation*}
\vert <x_{N(m,m)}-\widetilde{x},
\sum\limits_{k=1}^n y_k\Psi_k> \vert \leq
{(\sum\limits_{k=1}^n{[<x_{N(m,m)},\Psi_k> - \alpha_k]}^2)}^{\frac{1}{2}}\cdot
 \lVert f\lVert
\end{equation*}
Так как значение $n$ фиксировано, а значения
\begin{equation*}
<x_{N(m,m)},\Psi_K> \to \alpha_K$$ при
 $$m\to\infty 
\end{equation*}
то для достаточно больших $m$:
\begin{equation*}
\sum\limits_{k=1}^n{[<x_{N(m,m)},\Psi_k> - \alpha_k]}^2 < \epsilon^2
\end{equation*}
Таким образом
\begin{equation*}
\vert f(x_{N(m,m)}-\widetilde{x})\vert=
\vert <x_{N(m,m)}-\widetilde{x},f> \vert =
\vert (x_{N(m,m)}-\widetilde{x},y) \vert \leq
\epsilon [\lVert f\lVert + 2\epsilon]
\end{equation*}
для любого $f \in H^*$, если
 $\lbrace x_n\rbrace \in K$ при достаточно больших $m$, т.е. 
\begin{equation*}
x_{N(m,m)} \to \widetilde{x}
\end{equation*}
на множестве $K$, и множество $K$ слабо компактно в сепарабельном гильбертовом пространстве.\\

В доказательстве существенную роль играет теорема Рисса: $H=H^*$ и далее $(H^*)^*=H$. Банахово пространство $X$, для которого ${(X^*)}^*=X$ называется рефлексивным. В случае $X=L_p(T)$ можно показать, что общий вид линейного функционала определяется элементами $y \in L_q(T)$, 
$\frac{1}{p}+ \frac{1}{q}=1$:

\begin{equation*}
f(x)=<x,y>=\int\limits_T x(t)y(t)dt
\end{equation*}
(при $p=1$ пространство $L_\infty(T)$ - пространство измеримых и почти везде конечных\\ функций).\\\\
Ясно, что 
\begin{equation*}
{(L_p^*(T))}^*=L_p(T)
\end{equation*}
и пространство $L_p(T)$ рефлексивно.
\begin{theorem}
Верна общая теорема: условие ${\lVert x \lVert}_X \leq C$ для элементов множества $K \subset X$ является необходимым и достаточным условием слабой компактности множества $K$ в рефлексивных пространствах $X$.
\end{theorem}
В частности множество $K \in L_p(T)$ элементов, таких что 
\begin{equation*}
{\lVert x \lVert}_{L_p(T)} \leq C
\end{equation*}
слабо компактно.
\end{document}
